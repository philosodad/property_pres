
% $Header: /Users/paul/Classes/8220/Presentation/RCS/route-recovery.tex,v 1.1 2009/04/17 07:03:56 paul Exp $

\documentclass{beamer}

% This file is a solution template for:

% - Giving a talk on some subject.
% - The talk is between 15min and 45min long.
% - Style is ornate.



% Copyright 2004 by Till Tantau <tantau@users.sourceforge.net>.
%
% In principle, this file can be redistributed and/or modified under
% the terms of the GNU Public License, version 2.
%
% However, this file is supposed to be a template to be modified
% for your own needs. For this reason, if you use this file as a
% template and not specifically distribute it as part of a another
% package/program, I grant the extra permission to freely copy and
% modify this file as you see fit and even to delete this copyright
% notice. 


\mode<presentation>
{
  \usetheme{Frankfurt}
  % or ...
  
  \setbeamercovered{transparent}
  % or whatever (possibly just delete it)

  \usefonttheme[onlymath]{serif}
  
}

\usepackage{verbatim}
\usepackage{multirow} \usepackage{enumerate}
\usepackage{amsmath,enumerate} \usepackage{amsthm}
%\usepackage{algcompatible}
%\usepackage{algpseudocode}
\usepackage{algorithm}
%\usepackage{algorithmic}
%\usepackage{pstricks}
\usepackage{amssymb, latexsym}
\usepackage{xfrac}
\usepackage{mathtools}
\usepackage{graphicx}
%\usepackage[captionskip=5pt, nearskip=5pt, font=small]{subfig}
\DeclareGraphicsRule{*}{mps}{*}{}
%\usepackage{listings}

%pgfsettings
%\usepackage{pgf}
%\usepackage{tikz}
%\usetikzlibrary{decorations.pathmorphing} % LATEX and plain TEX when using Tik Z
%\usetikzlibrary{positioning}
%\tikzstyle{vx}=[draw,circle,fill=black!50,minimum size=2pt, inner sep=0pt, node distance=15mm]
%\tikzstyle{bup}=[decoration={bent, aspect=.3, amplitude=4}, decorate, ->, >=stealth]
%\tikzstyle{bdn}=[decoration={bent, aspect=.3, amplitude=-4}, decorate, ->, >=stealth]
%\tikzstyle{BUP}=[decoration={bent, aspect=.3, amplitude=8}, decorate, ->, >=stealth]
%\tikzstyle{BDN}=[decoration={bent, aspect=.3, amplitude=-8}, decorate, ->, >=stealth]
%\input{tikzamble.tex}
%\usepackage{algorithm, algorithmic}

\usepackage[english]{babel}
% or whatever

\usepackage[latin1]{inputenc}
% or whatever

\usepackage{times}
\usepackage[T1]{fontenc}
\usepackage{graphics}
% Or whatever. Note that the encoding and the font should match. If T1
% does not look nice, try deleting the line with the fontenc.

% le bib style
\bibliographystyle {IEEEtranS}
