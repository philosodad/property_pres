\begin{frame}
  \frametitle{Why Do We Test}
  \begin{block}{The Sixth Law of Software Design}
    The degree to which you know how your software behaves is the degree to which you have accurately tested it.

    - Max Kanat-Alexander, ``Code Simplicity''
  \end{block}
  \begin{itemize}
    \item Testing captures intent
    \item Testing verifies behavior
    \item Test cases define the edges of behavior 
  \end{itemize}
\end{frame}

\begin{frame}
  \frametitle{Typical Types of Testing}
  \begin{description}
    \item [Unit] Test behavior of individual functions
    \item [Integration] Test interaction of modules
    \item [Behavior] Test system inputs/outputs 
  \end{description}
  \begin{block}{Limitations of Testing}
    In all types of test, it is up to the test writer to determine the test cases.
  \end{block}
\end{frame}

\begin{frame}
  \frametitle{Good Practices in Testing}
  \begin{itemize}
    \item Avoid hard coded values
    \item Look for edge cases
    \item Make tests documentary
  \end{itemize}
  \begin{block}{Property Based Testing}
    Works by defining what the code should do and letting the system generate cases.
  \end{block}
\end{frame}
