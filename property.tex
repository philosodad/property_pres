
% $Header: /Users/paul/Classes/8220/Presentation/RCS/route-recovery.tex,v 1.1 2009/04/17 07:03:56 paul Exp $

\documentclass{beamer}

% This file is a solution template for:

% - Giving a talk on some subject.
% - The talk is between 15min and 45min long.
% - Style is ornate.



% Copyright 2004 by Till Tantau <tantau@users.sourceforge.net>.
%
% In principle, this file can be redistributed and/or modified under
% the terms of the GNU Public License, version 2.
%
% However, this file is supposed to be a template to be modified
% for your own needs. For this reason, if you use this file as a
% template and not specifically distribute it as part of a another
% package/program, I grant the extra permission to freely copy and
% modify this file as you see fit and even to delete this copyright
% notice. 


\mode<presentation>
{
  \usetheme{Frankfurt}
  % or ...
  
  \setbeamercovered{transparent}
  % or whatever (possibly just delete it)

  \usefonttheme[onlymath]{serif}
  
}

\usepackage{verbatim}
\usepackage{multirow} \usepackage{enumerate}
\usepackage{amsmath,enumerate} \usepackage{amsthm}
%\usepackage{algcompatible}
%\usepackage{algpseudocode}
\usepackage{algorithm}
%\usepackage{algorithmic}
%\usepackage{pstricks}
\usepackage{amssymb, latexsym}
\usepackage{xfrac}
\usepackage{mathtools}
\usepackage{graphicx}
%\usepackage[captionskip=5pt, nearskip=5pt, font=small]{subfig}
\DeclareGraphicsRule{*}{mps}{*}{}
%\usepackage{listings}

%pgfsettings
%\usepackage{pgf}
%\usepackage{tikz}
%\usetikzlibrary{decorations.pathmorphing} % LATEX and plain TEX when using Tik Z
%\usetikzlibrary{positioning}
%\tikzstyle{vx}=[draw,circle,fill=black!50,minimum size=2pt, inner sep=0pt, node distance=15mm]
%\tikzstyle{bup}=[decoration={bent, aspect=.3, amplitude=4}, decorate, ->, >=stealth]
%\tikzstyle{bdn}=[decoration={bent, aspect=.3, amplitude=-4}, decorate, ->, >=stealth]
%\tikzstyle{BUP}=[decoration={bent, aspect=.3, amplitude=8}, decorate, ->, >=stealth]
%\tikzstyle{BDN}=[decoration={bent, aspect=.3, amplitude=-8}, decorate, ->, >=stealth]
%\input{tikzamble.tex}
%\usepackage{algorithm, algorithmic}

\usepackage[english]{babel}
% or whatever

\usepackage[latin1]{inputenc}
% or whatever

\usepackage{times}
\usepackage[T1]{fontenc}
\usepackage{graphics}
% Or whatever. Note that the encoding and the font should match. If T1
% does not look nice, try deleting the line with the fontenc.

% le bib style
\bibliographystyle {IEEEtranS}


\title[] % (optional, use only with long paper titles)
{Proper-ly Testing Elixir}

\subtitle
{} % (optional)

\author[] % (optional, use only with lots of authors)
{Paul Daigle}
% - Use the \inst{?} command only if the authors have different
%   affiliation.

\institute[] % (optional, but mostly needed)
{ 
}

% - Use the \inst command only if there are several affiliations.
% - Keep it simple, no one is interested in your street address.

\date[] % (optional)
{2020-Aug-12}

\subject{}
% This is only inserted into the PDF information catalog. Can be left
% out. 



% If you have a file called "university-logo-filename.xxx", where xxx
% is a graphic format that can be processed by latex or pdflatex,
% resp., then you can add a logo as follows:

% \pgfdeclareimage[height=0.5cm]{university-logo}{university-logo-filename}
% \logo{\pgfuseimage{university-logo}}



% Delete this, if you do not want the table of contents to pop up at
% the beginning of each subsection:
\AtBeginSection[]
{
  \begin{frame}<beamer>
    \frametitle{Outline}
    \tableofcontents[currentsection,currentsubsection]
  \end{frame}
}


% If you wish to uncover everything in a step-wise fashion, uncomment
% the following command: 

%\beamerdefaultoverlayspecification{<+->}


\begin{document}

\begin{frame}
  \titlepage
\end{frame}

\begin{frame}
  \frametitle{Outline}
  \tableofcontents
  % You might wish to add the option [pausesections]
\end{frame}


\section{Testing}
\begin{frame}
  \frametitle{Why Do We Test}
  \begin{block}{The Sixth Law of Software Design}
    The degree to which you know how your software behaves is the degree to which you have accurately tested it.

    - Max Kanat-Alexander, ``Code Simplicity''
  \end{block}
  \begin{itemize}
    \item Testing captures intent
    \item Testing verifies behavior
    \item Test cases define the edges of behavior 
  \end{itemize}
\end{frame}

\begin{frame}
  \frametitle{Typical Types of Testing}
  \begin{description}
    \item [Unit] Test behavior of individual functions
    \item [Integration] Test interaction of modules
    \item [Behavior] Test system inputs/outputs 
  \end{description}
  \begin{block}{Limitations of Testing}
    In all types of test, it is up to the test writer to determine the test cases.
  \end{block}
\end{frame}

\begin{frame}
  \frametitle{Good Practices in Testing}
  \begin{itemize}
    \item Avoid hard coded values
    \item Look for edge cases
    \item Make tests documentary
  \end{itemize}
  \begin{block}{Property Based Testing}
    Works by defining what the code should do and letting the system generate cases.
  \end{block}
\end{frame}

\section{Property Based Testing}
\begin{frame}
  \frametitle{Parts of a Property Test}
  \begin{block}{Invariants}
    Defines what is always true about the output.

    Examples:
    \begin{itemize}
      \item In a sorted list, every item is <= the next item.
      \item An html document is enclosed by \texttt{<html><\textbackslash html>} tags.
    \end{itemize}
  \end{block}
  \begin{block}{Generators}
    Defines the edges of the input.

    Examples:
    \begin{itemize}
      \item a list of integers
      \item a valid http request
    \end {itemize}
  \end{block}
\end{frame}

\begin{frame}
  \frametitle{Property Testing Framework}
  A property testing framework should provide \emph{generators}, a way to create new generators, and run inputs based on those generators against defined \emph{invariants}.
\end{frame}

\section{Examples}
\begin{frame}
  \frametitle{Example Problem}
  \begin{block}{Partiphification}
    We have \emph{n} jobs to distribute over \emph{m} processes. Our goal is to write a function that partitions the jobs equally.
    \begin{enumerate}
      \item Jobs are input as an array
      \item We want the array split into subarrays
      \item The split should be as even as possible
      \item The function should be as generic as possible
    \end{enumerate}
  \end{block}
  \begin{block}{Invariants}
    \begin{itemize}
      \item We should have \emph{m} sublists
      \item Every job should be in a sublist
      \item Every job should appear once in all sublists
      \item The length of every sublist should be +-1 of any other
    \end{itemize}
  \end{block}
\end{frame}
\begin{frame}
  \frametitle{Example Code}
  It's time to look at some code!
\end{frame}

\section{Conclusions}
\begin{frame}
  \frametitle{General Thoughts}
  Property testing generates test cases for you, leaving you free to focus on the rules of your data transformations and code behavior.
  \begin{block}{Good Practices!}
    \begin{itemize}
      \item Avoid hard coded values (Generators)
      \item Look for edge cases (Generators)
      \item Make tests documentary (Invariants)
    \end{itemize}
  \end{block}
  \begin{block}{Challenges}
    \begin{itemize}
      \item Unfamiliar
      \item Complex generators
      \item Not a lot of guidance
    \end{itemize}
  \end{block}
\end{frame}
\begin{frame}
  \frametitle{Questions?}
  Twitter: @philosodad \\
  Twitch: @philosodad\\
  tech404: @philosodad
  github: github.com/philosodad\\
  medium: medium.com/perplexinomicon-of-philosodad\\


  Really, I'm pretty sure if you just search "philosodad" whatever comes back is probably me.
\end{frame}


\end{document}

